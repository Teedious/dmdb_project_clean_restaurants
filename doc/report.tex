\documentclass[conference]{IEEEtran}
%\IEEEoverridecommandlockouts
% The preceding line is only needed to identify funding in the first footnote. If that is unneeded, please comment it out.
\usepackage[english]{babel}
\usepackage[T1]{fontenc}
\usepackage[utf8]{inputenc}
\usepackage{lmodern}
\usepackage{microtype}
%\usepackage{cite}
\usepackage{amsmath,amssymb,amsfonts}
\usepackage{csquotes}
\usepackage{algorithmic}
\usepackage{graphicx}
\usepackage{textcomp}
\usepackage{xcolor}
\usepackage[style=ieee]{biblatex}
\usepackage{hyperref}
\bibliography{bibfile}

\def\BibTeX{{\rm B\kern-.05em{\sc i\kern-.025em b}\kern-.08em
    T\kern-.1667em\lower.7ex\hbox{E}\kern-.125emX}}
\begin{document}

\title{Cleaning and Visualizing a dirty set of restaurant data}


\author{\IEEEauthorblockN{Florian Loher}
\IEEEauthorblockA{\textit{Technical University of Applied
Science Regensburg} \\
florian.loher@st.oth-regensburg.de}
%\and
}

\maketitle

\begin{abstract}
This document shows a possible approach to cleaning an visualizing the dirty dataset provided at \url{https://hpi.de/naumann/projects/repeatability/datasets/restaurants-dataset.html}. It describes how the data is first audited, then cleaned in MongoDB, removing duplicates, using a common search engine to find correct restaurant names and standardizing road and city names. Lastly the data is visualized by generating a website that contains an OSM map and markers indicating the location of each restaurant.
\end{abstract}

\begin{IEEEkeywords}
MongoDB, Data cleaning
\end{IEEEkeywords}

\section{Introduction}
Data cleaning, also referred to as data scrubbing or data cleansing, is a research field concerned with improving the quality of faulty data. Typical aspects that are sought to be improved are the amount of duplicates, type errors or inconsistencies in the data\cite{Bilenko.2003}

%% Outline
\section{Basics}
\subsection{Data Cleaning Fundamentals}
\subsection{Stringmatching}


\printbibliography

\begin{thebibliography}{00}
\bibitem{Sarawagi} IEEE Data Eng. Bull., S. Sarawagi, ed., special issue on data
cleaning, vol. 23, no. 4, Dec. 2000.
\bibitem{b1} G. Eason, B. Noble, and I. N. Sneddon, \enquote{On certain integrals of Lipschitz-Hankel type involving products of Bessel functions}, Phil. Trans. Roy. Soc. London, vol. A247, pp. 529--551, April 1955.
\bibitem{b2} J. Clerk Maxwell, A Treatise on Electricity and Magnetism, 3rd ed., vol. 2. Oxford: Clarendon, 1892, pp.68--73.
\bibitem{b3} I. S. Jacobs and C. P. Bean, ``Fine particles, thin films and exchange anisotropy,'' in Magnetism, vol. III, G. T. Rado and H. Suhl, Eds. New York: Academic, 1963, pp. 271--350.
\bibitem{b4} K. Elissa, ``Title of paper if known,'' unpublished.
\bibitem{b5} R. Nicole, ``Title of paper with only first word capitalized,'' J. Name Stand. Abbrev., in press.
\bibitem{b6} Y. Yorozu, M. Hirano, K. Oka, and Y. Tagawa, ``Electron spectroscopy studies on magneto-optical media and plastic substrate interface,'' IEEE Transl. J. Magn. Japan, vol. 2, pp. 740--741, August 1987 [Digests 9th Annual Conf. Magnetics Japan, p. 301, 1982].
\bibitem{b7} M. Young, The Technical Writer's Handbook. Mill Valley, CA: University Science, 1989.
\end{thebibliography}
\end{document}